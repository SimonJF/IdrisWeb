\section{Dependently-typed Form Handling}

\label{form}
Programming web applications often involves processing user data, which may
then be used in further effectful computations. Data submitted using a form is
transmitted as a string as part of an HTTP request, which
traditionally involves losing associated type information.

This can in turn lead to risks; developers may assume that data is
of a certain type, and therefore discount the possibility that it may have been
modified by an attacker. One example would be the traversal of paginated data,
in which a form is used to make a request to retrieve the next page of data.
This may involve sending an integer detailing the current page, which could be
used in a query such as:

\begin{SaveVerbatim}{selectq}

'SELECT `name`, `address` FROM `staff` LIMIT ' + 
       page + ', 5';

\end{SaveVerbatim}
\useverb{selectq}

\noindent
The \texttt{page} variable is assumed to be an integer, but may instead be
modified by an attacker to include a malicious string which would alter the
semantics of the query, allowing an attacker to execute a blind SQL injection
attack. % Might be a good idea to cite an SQL injection paper which uses LIMIT
% clauses here

In this section, we present a DSL
for the creation of web forms which preserve type information, implemented
as a resource-dependent algebraic effect. Once the form has
been submitted, retrieved information is passed directly to a
developer-specified function for handling, without the need to manually check
and deserialise data. 

We begin with a simple example of a form which requests a user's name, and
echoes it back. Firstly, we define a form handler which echoes back a string
provided by the form handler. It has one argument of type \texttt{Maybe
String}, which accounts for the possibility that the user may have 
provided invalid data:

\begin{SaveVerbatim}{sayhello}

echo : Maybe String -> 
     FormHandler [CGI (InitialisedCGI TaskRunning)]
echo (Just name) = output ("Hello, " ++ name ++ "!")
echo _ = output "Error!"

\end{SaveVerbatim}
\useverb{sayhello}

\noindent
We then specify this in a list of handlers, detailing the arguments, available effects, handler function and unique identifier:

\begin{SaveVerbatim}{handerlist}

handlers : HandlerList
handlers = [handler args=[FormString], 
                    effects=[CgiEffect], 
                    fn=echo, 
                    name="echo"]

\end{SaveVerbatim}
\useverb{handerlist}

\noindent
We also define a form to take in a name from the user, and specify that it
should use the \texttt{echo} handler.

\begin{SaveVerbatim}{showhello}

showHelloForm : UserForm
showHelloForm = do
  addTextBox "Name" FormString Nothing
  useEffects [CgiEffect]
  addSubmit echo handlers

\end{SaveVerbatim}
\useverb{showhello}

\noindent
Finally, we specify that if data has been submitted for processing, then it
should be passed to the form handler. If not, then the form should be shown.

\begin{SaveVerbatim}{cgihello}

cgiHello : CGIProg [] ()
cgiHello = do
  handler_set <- isHandlerSet
  if handler_set then do
    handleForm handlers
    return ()
  else do
    addForm showHelloForm
    return ()

main : IO ()
main = runCGI [initCGIState] cgiHello

\end{SaveVerbatim}
\useverb{cgihello}

\noindent
When this CGI application is invoked, it will begin by outputting a form to the
page, requesting a name from the user. Upon submission of the form, the form
handler will be invoked, and the name will be used in the output.

In Sections ~\ref{formcons} and ~\ref{formhandling}, we examine implementation
of the form-handling system: namely, the effect which allows the creation of
forms, and the handling code which deserialises the data and passes it to the
user-specified handler function.  

\subsection{Form Construction}
\label{formcons}
Each form element is specified to hold a particular type of data, which, assuming that the correct type of data is specified by the user, is passed directly to the handler function. In order to encapsulate this, we firstly define the allowed data types as part of an algebraic data type, \texttt{FormTy}.

\begin{SaveVerbatim}{formty}

data FormTy = FormString | FormInt
            | FormBool   | FormFloat
            | FormList FormTy 

\end{SaveVerbatim}
\useverb{formty}

\noindent
Since types in \idris{} are first-class, we may use this to convert
between abstract and concrete representations of allowed form types:

\begin{SaveVerbatim}{interpformty}

interpFormTy : FormTy -> Type
interpFormTy FormString = String
interpFormTy FormInt = Int
interpFormTy FormBool = Bool
interpFormTy FormFloat = Float
interpFormTy (FormList a) = List (interpFormTy a)

\end{SaveVerbatim}
\useverb{interpformty}

\begin{SaveVerbatim}{formeff}
using (G : List FormTy, E : List WebEffect)
  data FormRes : List FormTy -> List WebEffect -> Type where
    FR : Nat -> List FormTy -> List WebEffect -> String -> FormRes G E
  
  data Form : Effect where
    AddTextBox      : (label : String) -> (fty : FormTy) -> (Maybe (interpFormTy fty)) -> 
                      Form (FormRes G E) (FormRes (fty :: G) E) () 
    AddSelectionBox : (label : String) -> (fty : FormTy) -> (vals : Vect m (interpFormTy fty)) -> 
                      (names : Vect m String) ->
                      Form (FormRes G E)  (FormRes (fty :: G) E) ()
    AddRadioGroup   : (label : String) -> (fty : FormTy) -> (vals : Vect m (interpFormTy fty)) ->
                      (names : Vect m String) -> (default : Int) ->
                      Form (FormRes G E)  (FormRes (fty :: G) E) ()
    AddCheckBoxes   : (label : String) -> (fty : FormTy) -> (vals : Vect m (interpFormTy fty)) ->
                      (names : Vect m String) -> (checked_boxes : Vect m Bool) ->
                      Form (FormRes G E)  (FormRes ((FormList fty) :: G) E) ()
    Submit          : (mkHandlerFn ((reverse G), E)) -> String -> 
                      Form (FormRes G E)  (FormRes [] [])       String
\end{SaveVerbatim}

\begin{figure*}[t]
\begin{center}
\useverb{formeff}
\end{center}
\caption{Form Effect}
\label{fig:formeffect}
\end{figure*}

\noindent
Again, we use \texttt{Effects} to build a form.
By recording
the type of each form element as it is added in the type of the form, we may
statically ensure that the user-supplied handler function is of the correct
type to handle the data supplied by the form: using an
incompatible handler will result in a compile-time type error. The \texttt{Form}
effect and associated resource \texttt{FormRes} is given in Figure 
\ref{fig:formeffect}.

The \texttt{using} notation indicates that where \texttt{G} and \texttt{E} occur within the block, they correspond to \emph{implicit arguments} within the type of the form construction resource. 
\texttt{G} is of type \texttt{List FormTy}, indicating the types of the elements within the form, and \texttt{E} is of type \texttt{List WebEffect}. This allows us to record the types of form elements, which we may later use to ensure that the type of the handler function corresponds to the types of the elements within the form.
%The \texttt{using} notation here indicates that within
%the block, where \texttt{G} and \texttt{E} occur they are implicit arguments with
%the given resource type.

The general process of form construction is illustrated by the \texttt{AddTextBox}
and \texttt{Submit} operations of the \texttt{Form} effect, shown in Figure ~\ref{fig:formeffect}. These use the resource associated with the effect, \texttt{FormRes}, to construct the form. Adding a field such as a text box adds a new type,
\texttt{fty} to the list of field types, carried in the resource. When the form
is complete, the \texttt{Submit} operation adds a submit button and returns the
HTML text for the form, flushing the list of field types, and using it to
construct the type for an appropriate handler function.
%
To specify a form instance, we define a function of type \texttt{UserForm}:

\begin{SaveVerbatim}{userform}

UserForm : Type
UserForm = EffM m [FORM (FormRes []) 
                        (FormRes [])] String
\end{SaveVerbatim}
\useverb{userform}

\noindent
%All forms are required to include a submit button, as mandated by the
%requirement that 
%
The input and output resource contains an empty list of types, which means that
any form which includes fields must also include a submit button. Adding fields
adds to the list of types, and only adding a submit button empties that list.
Note that there is no need to restrict this effect to running in the \texttt{IO}
monad since creating a form merely returns HTML text, with no side-effects by
default.

Before associating a handler function with the form, we must specify the
effects available to the handler. This is done with
\texttt{useEffects}, which updates the list of effects in the type of the form
resource. By doing this, we may subsequently use the effects in calculations at
the type level, in particular when calculating the type of the handler function
for the form. 

\begin{SaveVerbatim}{useeffs}

useEffects : (effs : List WebEffect) ->
             EffM m [FORM (FormRes G E)] 
                    [FORM (FormRes G effs)] ()
useEffects effs = (UseEffects effs)

\end{SaveVerbatim}
\useverb{useeffs}

\noindent
A \texttt{WebEffect} is an effect which is usable in a web application, and can
be converted to an \texttt{EFFECT} using:

\begin{SaveVerbatim}{interpwebeff}
webEffect : WebEffect -> EFFECT
\end{SaveVerbatim}
\useverb{interpwebeff}

While it is not possible to serialise arbitrary effects due to the associated
difficulties with serialising initial resource environments, we allow for three
effects to be serialised: \texttt{CGI}, \texttt{SQLITE} and \texttt{SESSION}.
This is, however, not an inherent limitation as the \texttt{Effects} library
permits introduction of additional effects within an effectful computation.

Handlers may only be associated with a form if they have argument types
corresponding to the types associated with the form elements. Additionally, we
wish to name the function in order for it to be serialised, whilst requiring a
proof that the specified name is associated with the function. If this were not
required, it would be possible to use a function satisfying the type
requirement, without guaranteeing that the serialised data corresponded to
that function.

We may specify a handler function of type \texttt{FormHandler}:

\begin{SaveVerbatim}{formhandler}

FormHandler : List EFFECT -> Type
FormHandler effs = Eff IO effs ()

\end{SaveVerbatim}
\useverb{formhandler}

\noindent
In order to associate a handler with a form, we call the \texttt{addSubmit}
function:

\begin{SaveVerbatim}{addsubmit}

addSubmit : (f :  mkHandlerFn ((reverse G), E)) ->
            (fns : HandlerList) ->
            {default tactics 
              { applyTactic findFn 100; solve; }
              prf : FnElem f fns} ->
            EffM m [FORM (FormRes G E)]
                   [FORM (FormRes [] [])] 
                   String
addSubmit f handlers {prf} = (Submit f name)
    where name : String
          name = getString' f handlers prf          

\end{SaveVerbatim}
\useverb{addsubmit}

\noindent
This function takes a handler function and a list of available handlers, along with
an automatically constructed proof (using the \texttt{default} argument) that the
handler is available.
Let us look at each aspect of this function in turn. Firstly, the
\texttt{mkHandlerFn} function calculates the required type of the handler
function from the list of types associated with the form elements, and the
effects we specified with \texttt{useEffects}. Note that since we prepend types
to the list of \texttt{FormTy}s as opposed to appending them, we must reverse
the list of form elements \texttt{G}.

\begin{SaveVerbatim}{mkhandlerfnty}

MkHandlerFnTy : Type
MkHandlerFnTy = (List FormTy, List WebEffect)

\end{SaveVerbatim}

\begin{SaveVerbatim}{mkhandlerfnp}

mkHandlerFn' : List FormTy -> List WebEffect -> Type
mkHandlerFn' [] effs = FormHandler (map webEffect effs) 
mkHandlerFn' (x :: xs) effs = Maybe (interpFormTy x) -> 
                              mkHandlerFn' xs effs 
\end{SaveVerbatim}

\begin{SaveVerbatim}{mkhandlerfn}
mkHandlerFn : MkHandlerFnTy -> Type 
mkHandlerFn (tys, effs) = mkHandlerFn' tys effs 

\end{SaveVerbatim}

\useverb{mkhandlerfnty}

\useverb{mkhandlerfnp}

\useverb{mkhandlerfn}

\noindent
The \texttt{mkHandlerFn} function takes a tuple describing the arguments and
web effects available to the handler function. When constructing the function
type, we wrap all arguments in a \texttt{Maybe}, in order to handle
failure should the supplied data fail to parse as the required type.
%
To store a reference to a handler function, we use the \texttt{HandlerFn} type:

\begin{SaveVerbatim}{handlerfn}

HandlerFn : Type
HandlerFn = (ft ** (mkHandlerFn ft, String))

\end{SaveVerbatim}
\useverb{handlerfn}

\noindent
The \texttt{**} notation denotes a dependent pair, in which the type of the second
element of the pair is parameterised over the value of the first element. It is
an existential binding:
the notation \texttt{(x ** P x)} can be read as ``there exists an \texttt{x} such that 
\texttt{P x} holds''.
Therefore a \texttt{HandlerFn} states that there exists a function type
\texttt{ft} such that we have a handler for it, and a unique string identifier
which is used to serialise a
reference to the handler function. 

In order to abstract away from this implementation detail, we make use of
\idris{} syntax rewriting rules. This allows us to define the following:

\noindent
\begin{SaveVerbatim}{syntaxhandler}

syntax 
  "handler args=" [args] ", effects=" [effs] ", fn=" [fn] 
  ", name=" [name] = ((args, effs) ** (fn, name))

\end{SaveVerbatim}
\useverb{syntaxhandler}

\noindent
We may then define handlers in a more intuitive fashion, without being
concerned with the implementation details. This allows us to write a handler
with one String argument, making use of the CGI effect, associated with the
\texttt{echo} handler function as follows:

\begin{SaveVerbatim}{handlerex}

handler args=[FormString], 
        effects=[CgiEffect], 
        fn=echo, 
        name="echo"

\end{SaveVerbatim}
\useverb{handlerex}

\noindent
We then store each \texttt{HandlerFn} in a \texttt{HandlerList}.

\begin{SaveVerbatim}{handlerlist}

HandlerList : Type
HandlerList = List HandlerFn

\end{SaveVerbatim}
\useverb{handlerlist}

To enforce the requirement that a supplied handler function must be in the
list of available handlers, and therefore allow us to retrieve the name with
which to serialise the handler, we require a \textit{list membership proof},
\texttt{FnElem f fns}, which statically guarantees that a given item resides in
a list.

\begin{SaveVerbatim}{fnelem}

using (xs : HanderList , f : mkHandler (reverse G, E))
  data FnElem : mkHandlerFn ((reverse G), E) -> 
                HandlerList -> Type where
       FnHere  : FnElem f (((reverse G, E) ** 
                                  (f, fStr)) :: xs)
       FnThere : FnElem f xs -> FnElem f (x :: xs)
       
\end{SaveVerbatim}
\useverb{fnelem}

\noindent
\texttt{FnElem} is parameterised over \texttt{G} and \texttt{E}, the types of the
form elements and the effects used by the handler function. \texttt{FnHere}
is a proof that the element is at the head of the current point of the list,
whereas \texttt{FnThere} is a proof that the element is in the tail of 
the list. 
We then use an automatic proof search to
generate the proof at compile time, should one exist. The proof
may then be used in subsequent computations: we use it to retrieve
the unique identifier for the function. If the automated proof search fails,
compilation will fail.

Finally, we serialise the argument types, supported effects, and return
type of the handler, to allow the form data to be 
deserialised and ensure that the correct handler is executed on the
server. 

Although sending details of the handler function to the client may appear to be
a security risk, we envisage that the use of symmetric encryption or a
cryptographic nonce would mitigate this. Ultimately, we hope to implement a
web server with persistent state, which would eliminate the need for
serialisation altogether.

Running form construction is achieved as an operation of the \texttt{CGI}
effect, \texttt{AddForm}, which then outputs the generated HTML to the page.
The generated metadata describing the handler function is serialised as a
hidden HTML field.

\subsection{Form Handling}
\label{formhandling}
Once the form has been submitted, a web application may handle the submitted
data by invoking \texttt{HandleForm}. This will check for the
existence of the hidden \texttt{handler} field, which contains the previously
serialised metadata about the form handler, before deserialising the
data into a \texttt{MkHandlerFnTy}. 

With this data, we then look up the function in the list of registered
handlers by using the unique handler identifier. In order to apply
the handler function to the data submitted in the form, we must first prove to
the type checker that the deserialised \texttt{MkHandlerFnTy} is the same as
the one retrieved from the list of registered handlers. We do this by making
use of the \texttt{decEq} function, which determines whether two types are
equal, returning a proof of equality if so, and a proof of inequality if not.

\begin{SaveVerbatim}{deceqty}

decEq : DecEq t => (x : t) -> (y : t) -> Dec (x = y)

\end{SaveVerbatim}
\useverb{deceqty}

We then use the \texttt{with} construct, inspired by
\textit{views} in Epigram \cite{mcbride.mckinna:viewfromleft}, to rewrite the
arguments on the left hand side. This allows us to construct a function which,
given the stored handler, the data required to
construct the function type and the \texttt{MkHandlerFnTy} deserialised from
the form, determines whether the two \texttt{MkHandlerFnTy}s are decidably
equal. If so, we rewrite this on the left hand side since the equality proof
demonstrates that the recorded function may also be used to handle the form
data. If not, the computation fails.

\begin{SaveVerbatim}{checkfunctionsty}

checkFunctions : (reg_fn_ty : MkHandlerFnTy) -> 
                 (frm_fn_ty : MkHandlerFnTy) -> 
                 mkHandlerFn reg_fn_ty -> 
                 Maybe (mkHandlerFn frm_fn_ty)
\end{SaveVerbatim}

\begin{SaveVerbatim}{checkfunctions}
checkFunctions reg_ty frm_ty reg_fn with 
                             (decEq reg_ty frm_ty)
  checkFunctions frm_ty frm_ty reg_fn 
                        | Yes refl = Just reg_fn
  checkFunctions reg_ty frm_ty reg_fn 
                        | No _ = Nothing

\end{SaveVerbatim}

\useverb{checkfunctionsty}

\useverb{checkfunctions}

\noindent
We may then parse the arguments according to the types specified by the handler
function, and then apply the arguments to the handler function.
Finally, we may run the handler function, ensuring that
all updates made to the CGI state are propagated.

%-----------------------------
%-----------------------------