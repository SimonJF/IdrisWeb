\section{Related Work}
\citet{meijer:cgi} implemented a CGI library which was among the first
libraries to handle web scripting monadically, and allows the user to implement
application logic without having to consider the low-level details such as
parsing in CGI data from the environment, or printing headers to the remote
browser. The library also provides support for cookies and basic form handling. 

\citet{thiemann:wash} adds the notion of a CGI Session for maintaining
state, and provides more sophisticated form-handling methods. In particular,
callbacks may be associated with submit buttons, with nameless representations
for form inputs. Due to the unavailability of full dependent types in Haskell,
however, this implementation does not statically verify the suitability of the
callback function for the form inputs. Both implementations of the CGI library,
being built upon monads, mean that the use of additional effects such as
database access is achieved either through monad transformers or 
performing arbitrary IO operations. Both of these approaches are limited---the
former does not scale well to multiple effects, and the latter allows for the
introduction of errors by allowing the violation of resource usage protocols.

\citet{plasmeijer:idata} describe an alternative approach to type-safe form
handling through the
\textit{i}nteractive \textit{Data}, or \idata{} abstraction. Instead of
processing being triggered by form submission, as in the approach described in
this paper, applications created in the \idata{} toolkit are 
edit-driven. This means that upon a component being edited, a computation
occurs, given the state of the current form. This is saved for
future computations. Should a user enter invalid data, for example by entering
text in a field designated for integers, the change will be reverted. This is
demonstrated practically through the use of \idata{} to implement a conference
management system \cite{plasmeijer:cms}.

The concept of \idata{} is taken further by the introduction of \itasks{}
\cite{plasmeijer:itasks}, which make use of a workflow system to allow multiple
\idata{} forms to interact with one another. This is achieved using
high-level combinators which allow the implementation of concepts such as
recursion, sequence and choice in a scalable fashion. These high-level
abstractions are elegant, but the style and syntax differ substantially from a
traditional web application. Our approach takes the concept of type-safe input
handling and uses it in a more traditional fashion, whilst retaining the
type-retention guarantees afforded by \idata{} elements.

Ur\/Web \cite{urweb} is a library built for the Ur language, which does not use
full dependent types but does have an expressive type system with 
record types and type-level computation.  By using these concepts,
Ur\/Web may generate provably correct and unexploitable DOM code and SQL
queries from records, without requiring developers to supply proofs.  In
contrast to using runtime code generation, which is prone to obscure code
generation errors, Ur\/Web makes use of its static type system to guarantee
that metaprograms---in this case, generated SQL and DOM code---must be correct
and secure.  Such ideas regarding the use of static checking of metaprogram
generation will be extremely useful when considering an object-relational
mapping system, which we hope to implement in the near future. It will also be
interesting to see how such concepts may be applied with a yet more expressive
type system involving full dependent types.

%Ur\/Web \cite{urweb} is a library for the Ur language which makes use of
%dependent types to enable the generation of provably-correct and secure queries
%and DOM code, without requiring overly complex types or proofs. Through the use
%of the Ur\/Web DSL, developers may generate well-typed code from records. By
%statically verifying the \textit{code generator}, it is therefore possible to
%assert that the generated code is secure. Such ideas are extremely interesting
%for the implementation of object-relational mapping, which we hope to implement
%in the near future.

WebDSL \cite{webdsl} is a domain-specific language written primarily to
introduce new abstractions which aim to reduce the amount of boilerplate code
that must be written and maintained by developeres. WebDSL is built on top of
Java, which often includes a large amount of redundant code such as accessor
and mutator functions within entity classes. The Java Persistence API (JPA)
\cite{jpa} provides an object-relational mapping through the use of Java 5
annotations, which may then be used to construct database tables. These
annotations soon become complex, however, and coupled with redundant
boilerplate code, data model declarations may soon become unwieldy. Through the
use of WebDSL, these data model declarations can be much more compactly
declared, and elaborated into Java code by parsing the data-modelling DSL into
an abstract syntax tree, applying rewrite rules, and pretty-printing. WebDSL
also applies similar concepts to implement a \textit{template system} for such
objects, which allows the data to be used in code generation. We look to
implement many of these ideas, but as effects within the IdrisWeb
framework, as with the form construction effect.

%-----------------------------
%-----------------------------

\section{Conclusions}

Dependently-typed languages promise to support machine checkable program
correctness proofs, but to date they have remained relatively unused for
practical purposes. By using embedded domain-specific languages, we can
abstract away some of the complexities of creating correctness proofs and
provide expressive libraries, giving guarantees by the successful compilation
of a program (assuming the use of specific enough types) without additional
proofs.

Our framework provides several static
guarantees. Data submitted by users is inherently unsafe and
means systems are vulnerable to attacks
such as SQL injection. This particular threat is ameliorated due
to elements being associated with specific types during form construction. This
immediately eliminates the possibilities of SQL injection attacks on non-string
types. Since failures are handled transparently, no runtime errors are output
to the browser, meaning that attackers may not use such information to aid
attacks. Additionally, since checking is performed on the types of the form
elements and the types of arguments accepted by the handler, it is impossible
to associate a form with a handler incompatible with the submitted data.

Many external libraries also follow (unchecked or dynamically checked) 
resource usage protocols.
Incorrect usage is however still possible, for
example by forgetting to release acquired resources or failing to
initialise a library correctly. By creating high-level bindings to these
libraries, however, we may statically enforce these resource-usage protocols,
ensuring that the libraries are used correctly. Whilst previous work has
demonstrated that this is possible through the use of embedded DSLs
\cite{brady:edsl} and dependent algebraic effects \cite{brady:effects},
this paper has provided more substantial examples of real-world applications. 

In particular, the framework guarantees that it is not possible for a CGI 
application to
produce an internal server error due to content being written to the remote
host prior to headers. With regard to database access, we may statically
guarantee that library calls are made in the correct order, and calls to
retrieve rows of data are made only when more data is available. Additionally, by encoding desired invariants within operation types, we may gain static guarantees about adherence to resource usage protocols and failure handling.
Enforcing resource usage protocols also guards against common programmer
errors, saving debugging time by identifying errors at compile time.

\subsection{Further Work}

We have shown that embedded domain-specific languages using dependent types and
algebraic effects can be used to increase confidence in web applications by
providing additional static guarantees about runtime behaviour, but much more
can be done using the same approach.

There are many other applications which make use of specific resource usage
protocols, for example popular libraries such as
\textit{libgcrypt}\footnote{\texttt{http://directory.fsf.org/wiki/Libgcrypt}}.
Applying a similar approach would allow for sensitive programs requiring
cryptographic routines to be written using a language with full dependent
types, in turn adding an extra layer of confidence in their security. 

Whilst the use of CGI allows for experimenting with the use of dependent types
in a real-world scenario such as web programming, there remain practical
considerations about its scalability, as a separate process must be created for
each individual request. We believe that the use of FastCGI may alleviate this,
but ultimately, we would like to create a web server written in \idris{}, which
would make more efficient usage of resources. 

Since at this stage we have concentrated on the use of dependent types for
enforcing resource usage protocols and type-safe form handling, we currently
handle the generation of HTML in an unstructured manner. Future work will
entail a DOM library to facilitate the generation and manipulation of HTML, in
turn giving stronger guarantees about its correctness. Other planned features
include a template system, allowing for web pages to be automatically generated
from data, and an object-relational mapping system which will allow users to
manipulate records which can then be automatically written to the database,
instead of having to update tables manually via SQL queries.

Type providers, as originally implemented in F\# \cite{msr:tp}, are an
interesting method by which external data sources may be used to import
external information, such that it may be used during compilation. 
In this way, it becomes possible to use the extra type
information to statically ensure the validity of artefacts such as SQL
queries and data structures. If data structures within the program do not
conform to a given database schema, for example, then the program will not
type-check.  This has been implemented for 
\idris{} 
\cite{christiansen:dtp}, exploiting the fact that types can be
calculated by functions to avoid generating extra code in the type provider
step. Dependent type providers additionally have stronger safety guarantees as
they may not generate unchecked code, but at the same time this is matched by a
decrease in expressiveness. Nonetheless, such techniques provide a promising
mechanism to verify the semantic soundness of programs and we look to
investigate their integration in further work.

Dependently-typed languages provide great promise for the construction of
secure and correct programs. Through the use of embedded domain-specific
languages, we hope that more developers may benefit from the extra guarantees
afforded by dependent types, resulting in more stable, secure applications.

% Cryptography bindings would be really good.
% Integration with a web server instead of doing everything over CGI
% Improvements to form handling system
% Less raw SQL, use get more type-safety by using more complex EDSLs for database access thus further minimising errors