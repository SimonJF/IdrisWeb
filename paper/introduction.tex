\section{Introduction}

Web applications, whilst ubiquitous, are also prone to incorrect construction
and security exploits such as SQL injection \cite{owasp:sqli} or cross-site
scripting \cite{owasp:xss}. Security breaches are
far-reaching, and high profile cases involve large corporations such as Sony,
who suffered a well-publicised and extremely costly SQL injection breach in
2011 \cite{ieee:sony}, and \textit{Yahoo!}, who suffered a breach in 2012
\cite{imperva:yahoo}. 

Many web applications are written in dynamically-checked scripting languages
such as PHP, Ruby or Python, which facilitate rapid development
\cite{w3techs:webpls}. Such languages do not, however, provide the same static 
guarantees about runtime behaviour afforded by
programs with more expressive, static type systems, instead relying on
extensive unit testing to ensure correctness and security. 

%%------- EXAMPLE
Let us consider a simple database access routine, written in
PHP, where we wish to obtain the name and address of every employee working in
a given department, \texttt{\$dept}. We firstly construct an object
representing a database connection, where the arguments are the database host,
user, password and name respectively:

\begin{SaveVerbatim}{conn}

$conn = new mysqli("localhost", "username", 
                   "password", "db");

\end{SaveVerbatim}
\useverb{conn}

\noindent
We should then check to see if the connection was successful, and exit
if not:
%This check is optional, so
%it would be possible to omit it. However, this would cause
%problems with later steps.

\begin{SaveVerbatim}{connerr}

if (mysqli_connect_errno()) { exit(); }

\end{SaveVerbatim}
\useverb{connerr}

\noindent
We then create a prepared statement detailing our query, and bind the
\texttt{`dept'} value:

\begin{SaveVerbatim}{prepare}

  $stmt = $conn->prepare("SELECT `name`, `address` 
     FROM `staff` WHERE `dept` = ?);
  $stmt->bind_param('s', $dept);

\end{SaveVerbatim}
\useverb{prepare}

\noindent
After the parameters have been bound, we execute the statement, assign
variables into which results will be stored, and fetch each row in turn. 
Failure to execute a statement before attempting to fetch rows would cause an error, as would attempting to execute a statement without binding variables to it.

\begin{SaveVerbatim}{stmtexec}

$stmt->execute();
$stmt->bind_result($name, $address);
while ($stmt->fetch()) {
  printf("Name: %s, Age: %s", $name, $age);
}

\end{SaveVerbatim}
\useverb{stmtexec}

\noindent
Finally, once the statement and connection are no longer needed, they should be
closed in order to discard the associated resources:

\begin{SaveVerbatim}{connclose}

  $stmt->close();
  $conn->close();

\end{SaveVerbatim}
\useverb{connclose}

\noindent
Even in this small example, there exists a precise resource usage protocol
which must be followed for successful and robust operation.
Firstly, a connection to the database must be opened. The object-oriented style
used in the example encapsulates this to an extent, as the object must be
created in order for operations to be performed, however it is less obvious in
a procedural version of the code. Secondly, a prepared statement is created,
using the raw SQL and placeholders to which variables are later bound. The
statement is then executed, and each row is retrieved from the database.
Finally, the resources are freed. 

Problems may arise if the protocol is not followed correctly.
A developer may, for example, accidentally close a statement whilst still
retrieving rows, which would cause a runtime error. Similarly, a developer may
omit closing the statement or connection, which can lead to
problems such as resource leaks in longer-running server applications.
However, in conventional programming languages, there is no way to check
automatically, at compile-time, that a protocol is followed.

In contrast, the use of \textit{dependent types} makes it possible
to specify a program's behaviour precisely, and to check that a 
specification is followed.
%
Unfortunately, automatic verification by a compiler can be difficult or
often impossible, requiring additional proofs to be given by the developer.

This complexity can be addressed through the use of \textit{embedded
domain-specific languages} (EDSLs) to abstract away the complexity of the
underlying type system. Embedding a domain-specific language in
a dependently typed host language
allows domain experts to write 
domain-specific code, with the EDSL itself used to provide the proof that the
code is correct.

\idris{} \cite{brady2013idris} is a language with full dependent types, and
extensive support for EDSLs through overloading and syntax macros. Through the
use of \idris{}, and a framework for describing resource protocols using
\emph{algebraic effects}~\cite{brady:effects}, we
present a dependently-typed web framework allowing the construction of
programs with additional guarantees about correctness and security, whilst
minimising the increase in development complexity. 

\subsection{Contributions}
The primary contribution of this paper is the application of 
dependent types to provide strong static guarantees
about the correctness and security of web applications, whilst minimising
additional development complexity. In particular, we present:

\begin{itemize}
\item Representations of CGI, Databases and sessions as
\textit{resource-dependent algebraic effects}, allowing programs to be accepted
only when they follow clearly defined resource usage protocols. 
(Section~\ref{rup})

\item Type-safe form handling, preserving type information and managing
user input, therefore increasing applications' resilience to attacks such as
SQL injection and cross-site scripting. (Section~\ref{form})

\item An extended example: a message board application, demonstrating the usage
of the framework in practice. (Section~\ref{effects})

\end{itemize}

We achieve these without extending the host language. Every
resource protocol we implement is pure \Idris{}, using a library for
resource-dependent algebraic effects~\cite{brady:effects} and \Idris{}'
features for supporting domain-specific language implementation such as
syntax macros and overloading. In particular, this means the same techniques
can be applied to other resources, and most importantly, combinations
of resources and DSLs implemented in this way are \emph{composable}.

%We structure the remainder of this paper as follows. We provide a brief
%overview of the \texttt{Effects} framework in Section ~\ref{effects}; explain
%how this may be used to ensure adherence to resource usage protocols for CGI,
%SQLite and a session handler in Section ~\ref{rup}; describe an EDSL for
%type-safe form handling in Section ~\ref{form}, implemented using
%\texttt{Effects}; and discuss the larger example
%of a message board system making use of these components in Section
%~\ref{messageboard}.

The code used to implement the framework and all associated examples used in
this paper is available online at
\url{http://www.github.com/idris-hackers/IdrisWeb}.

% =================================================
% =================================================