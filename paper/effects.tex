\section{An overview of the \texttt{Effects} framework}
\label{effects}

\texttt{Effects}~\cite{brady:effects} is an \idris{} library which handles
side-effects such as state, exceptions, and I/O as \emph{algebraic
effects}~\cite{Plotkin2009}. In particular, it supports parameterising effects
by an input and output \emph{state}, which permits effectful programs to track
the progress of a resource usage protocol. Effectful programs are written
in a monadic style, with \texttt{do}-notation, with their type stating which
specific effects are available.
Effectful
programs are described using the following data type,
in the simplest case:

\begin{SaveVerbatim}{efftype}

Eff : (m : Type -> Type) ->
      (es : List EFFECT) -> (a : Type) -> Type

\end{SaveVerbatim}
\useverb{efftype}

\noindent
\texttt{Eff} is parameterised over a \emph{computation context}, \texttt{m}, 
which describes the context in which the effectful program will be run, a
list of side effects \texttt{es} that the program is permitted to use, and the
program'’s return type \texttt{a}. The name \texttt{m} for the computation context is
suggestive of a monad, but there is no requirement for it to be so.

For example, the following type carries an integer state,
throws an exception of type \texttt{String} if the state reaches 100, 
and runs in a \texttt{Maybe} context:

\begin{SaveVerbatim}{addState}

addState : Eff Maybe [STATE Int, EXCEPTION String] ()
addState = do val <- get
              when (val == 100) (raise "State too big")
              put (val + 1)

\end{SaveVerbatim}
\useverb{addState}

\subsection{Implementing Effects}

Effects such as `State' and `Exception' are described as algebraic data types,
and run by giving \emph{handlers} for specific computation contexts. 
Effects have a corresponding \emph{resource} (in the case of state, the
resource is simply the current state). Executing an effectful operation may
change the resource and return a value:

\begin{SaveVerbatim}{efffam}

Effect : Type
Effect = (in_res : Type) -> (out_res : Type) -> 
         (val : Type) -> Type

\end{SaveVerbatim}
\useverb{efffam}

\noindent
For example, the state effect is described as follows:

\begin{SaveVerbatim}{stateeff}

data State : Effect where
     Get :      State a a a
     Put : b -> State a b ()

\end{SaveVerbatim}
\useverb{stateeff}

\noindent
That is, \texttt{Get} returns a value of type \texttt{a} without updating
the resource type. \texttt{Put} returns nothing, but has the effect of updating
the resource. To make an effect usable, we implement a handler
for a computation context by making an instance of the following class:

\begin{SaveVerbatim}{effhandler}

class Handler (e : Effect) (m : Type -> Type) where
     handle : res -> (eff : e res res' t) -> 
              (k: res' -> t -> m a) -> m a

\end{SaveVerbatim}
\useverb{effhandler}

\noindent
The \texttt{handle} function takes the input resource, an effect which may
update that resource and execute a side-effect, and a continuation \texttt{k}
which takes the updated resource and the return value of the effect. We use
a continuation here primarily because there is no restriction on the number of
times a handler may invoke the continuation (raising an exception, for example,
will not invoke it). Reading and updating states is handled
for all computation contexts \texttt{m}:

\begin{SaveVerbatim}{statehandler}

instance Handler State m where
     handle st Get     k = k st st
     handle st (Put n) k = k n ()

\end{SaveVerbatim}
\useverb{statehandler}

\noindent
Finally, we promote \texttt{State} into a concrete effect \texttt{STATE}, and
the \texttt{Get} and \texttt{Put} operations into functions in \texttt{Eff}, as
follows:

\begin{SaveVerbatim}{stateconc}

STATE : Type -> EFFECT
STATE t = MkEff t State

get : Eff m [STATE x] x
get = Get 

put : x -> Eff m [STATE x] ()
put val = Put val

\end{SaveVerbatim}
\useverb{stateconc}

\noindent
A concrete effect is simply an algebraic effect type paired with its current
resource type. This, and other
technical details, are explained in full elsewhere~\cite{brady:effects}.
For the purposes of this paper, it suffices to know how to describe and
handle new algebraic effects.

\subsection{Resource Protocols as Effects}
\label{rp-effects}
\begin{SaveVerbatim}{fileeff}
{-                                             { Input resource type }   { Output resource type } { Value } -}
 
data FileIO : Effect where
     Open      : String -> (m : Mode) -> FileIO     ()                    (Either () (OpenFile m)) ()
     Close     :                         FileIO     (OpenFile m)          ()                       ()

     ReadLine  :                         FileIO     (OpenFile Read)       (OpenFile Read)          String
     WriteLine : String ->               FileIO     (OpenFile Write)      (OpenFile Write)         ()
     EOF       :                         FileIO     (OpenFile Read)       (OpenFile Read)          Bool
\end{SaveVerbatim}

\begin{figure*}[t]
\begin{center}
\useverb{fileeff}
\end{center}
\caption{File Protocol Effect}
\label{fig:fileeffect}
\end{figure*}
More generally, a program might modify the set of effects available.
This might be desirable for several reasons, such as adding a new
effect, or to update an index of a dependently typed state. In this
case, we describe programs using the \texttt{EffM} data type:

\begin{SaveVerbatim}{effmtype}

EffM : (m : Type -> Type) ->
       (es : List EFFECT) -> (es' : List EFFECT) ->
       (a : Type) -> Type

\end{SaveVerbatim}
\useverb{effmtype}

\noindent
\texttt{EffM} is parameterised over the context and type as before, but
separates input effects (\texttt{es}) from output effects (\texttt{es'}). 
In fact, \texttt{Eff}
is defined in terms of \texttt{EffM}, with equal input/output effects.
We can use this to describe how effects follow resource protocols. A simple
example is a file access protocol, where a file must be opened before it
is read or written, and a file must be closed on exit. 

Figure \ref{fig:fileeffect} shows how the protocol is encoded as an
effect.
Note that the types of the input and output resources describes how resource
state changes in each operation: opening a file may fail, so
changes an empty resource to
a resource containing either a unit or an open file; 
reading a line is only possible if the
resource is a file open for reading, etc.
The handler for this effect for an \texttt{IO} computation context will
execute the required primitive I/O actions.

The following program type checks, and therefore implicitly carries 
a proof that the file resource protocol is followed correctly:

\begin{SaveVerbatim}{testfile}

testFile : Eff IO [FILE_IO (), STDIO] () 
testFile = do open "testFile" Read
              if_valid then do str <- readLine
                            close
                            putStrLn str
                       else putStrLn "Error"

\end{SaveVerbatim}
\useverb{testfile}

\noindent
The type of \texttt{testFile} states
that File I/O and console I/O are available effects, and in particular that
the resource associated with the File I/O will be in the same state on entry
and exit. We use \texttt{if\_valid} to handle possible failure---this is
a function provided by the \texttt{Effects} library which checks whether
a resource of type \texttt{Either a b} indicates failure (\texttt{Left a})
or success (\texttt{Right b}). In this case, the initial resource in the valid
branch is \texttt{OpenFile Read}, indicating that the file has been successfully 
opened for reading, whereas in the failing branch, the resource remains as \texttt{()}.

Attempting to write to the file, failing to check for
success, or failing to open or close the
file, would cause a \emph{compile-time} error. 

We will use this technique extensively
throughout this paper: describe a resource usage protocol in terms of
state transitions; implement an effect which captures that protocol; and implement
programs which, by using this effect, implicitly carry a proof that the resource
protocol has been correctly followed.


% =================================================
% =================================================