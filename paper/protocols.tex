\section{Modelling resource usage protocols}
\label{rup}
In this section, we show how three effects 
(CGI, database access and a simple session handler) may be implemented, and
describe the benefits of developing programs using this technique over simply
handling them in \texttt{IO} or as monad transformers.

% =================================================

\subsection{CGI}
%TODO: Fix up the labels
\begin{figure}[htpb!]
\centering
\scalebox{0.8}{
\begin{tikzpicture}[>=latex]
  \tikzstyle{state} = [draw, very thick, fill=white, rectangle, minimum height=3em, minimum width=7em, node distance=6em, font={\sffamily\bfseries}]
  
  \tikzstyle{stateEdgePortion} = [black,thick];
  \tikzstyle{stateEdge} = [stateEdgePortion,->];
  \tikzstyle{edgeLabel} = [pos=0.5, font={\sffamily\small}];

  \node[initial,state] (A)              {Uninitialised};
  \node[state]         (B) [below of=A] {Initialised};
  \node[state]         (C) [below of=B] {TaskRunning};
  \node[state]         (D) [below of=C] {TaskCompleted};
  \node[state]         (E) [below of=D] {HeadersWritten};
  \node[state]         (F) [below of=E] {ContentWritten};

  \path (A) edge[stateEdge]   node[edgeLabel, xshift=3em] {\texttt{initialise}} (B)
        (B) edge[stateEdge]   node[edgeLabel, xshift=3em] {\texttt{startTask}} (C)
        (C) edge[stateEdge]   node[edgeLabel, xshift=3em] {\texttt{finishTask}} (D)
        (D) edge[stateEdge]   node[edgeLabel, xshift=3.5em] {\texttt{writeHeaders}} (E)
        (E) edge[stateEdge]   node[edgeLabel, xshift=3.5em] {\texttt{writeContent}} (F);
\end{tikzpicture}
}
\caption{CGI States}
\label{fig:cgistates}
\end{figure}

The Common Gateway Interface (CGI) is used to invoke an application on a web server, making use of environment
variables to convey information gained from an HTTP request and using standard
output to communicate with the remote client. Importantly, HTTP headers must be
correctly written to the browser prior to any other output; failure to do so
will result in an HTTP Error 500.

By modelling CGI as a resource-dependent algebraic effect, we may enforce a
resource usage protocol which
prevents arbitrary IO from being performed and therefore
ensures that the headers are written correctly. We
define an effect, \texttt{Cgi}, and an associated resource,
\texttt{InitialisedCGI}, parameterised over the current state,
\texttt{CGIStep}, and containing a
\texttt{CGIInfo} record which contains information from the request. We
represent an uninitialised CGI process as the unit type, \texttt{()}.

\begin{SaveVerbatim}{cgistep}

data CGIStep = Initialised   | TaskRunning 
             | TaskCompleted | HeadersWritten 
             | ContentWritten

data InitialisedCGI : CGIStep -> Type where
     ICgi : CGIInfo -> InitialisedCGI s

\end{SaveVerbatim}
\useverb{cgistep}

\noindent
Figure~\ref{fig:cgistates} shows the states through which the CGI program
progresses, and Figure~\ref{fig:cgieffect} shows how this is represented
as a resource-dependent algebraic effect. Each operation performed in an effectful
program requires the resource to be of a certain type, and the completion of
the operation may alter the type or value of the resource. The \texttt{Cgi}
effect declaration shows these resource updates in the types of each operation,
effectively specifying a state machine.

Upon creation, the CGI application is uninitialised, meaning that environment variables have not been queried to populate the CGI state. The only operation
that can be performed in this state is initialisation: by calling
\texttt{initialise}, a \texttt{CGIInfo} record is populated, and the state transitions
to \texttt{Initialised}. The \texttt{Init} operation is defined as part of the
\texttt{Cgi} effect, and involves transitioning from the uninitialised state to
the initialised state.

\begin{SaveVerbatim}{cgieff}
{-                     { Input resource type }         { Output resource type }        { Value } -}

data Cgi : Effect where
    Init         : Cgi ()                              (InitialisedCGI Initialised)    ()
    StartRun     : Cgi (InitialisedCGI Initialised)    (InitialisedCGI TaskRunning)    ()
    FinishRun    : Cgi (InitialisedCGI TaskRunning)    (InitialisedCGI TaskCompleted)  ()
    WriteHeaders : Cgi (InitialisedCGI TaskCompleted)  (InitialisedCGI HeadersWritten) ()
    WriteContent : Cgi (InitialisedCGI HeadersWritten) (InitialisedCGI ContentWritten) ()
    OutputData   : String -> 
                   Cgi (InitialisedCGI TaskRunning)    (InitialisedCGI TaskRunning)    ()
    RunAction    : Env IO (CGI (InitialisedCGI TaskRunning) :: effs) -> CGIProg effs a -> 
                   Cgi (InitialisedCGI TaskRunning)    (InitialisedCGI TaskRunning)    a
\end{SaveVerbatim}

\begin{figure*}[t]
\begin{center}
\useverb{cgieff}
\end{center}
\caption{CGI Effect}
\label{fig:cgieffect}
\end{figure*}

Additional operations, including those to query \texttt{POST} and \texttt{GET}
variables, are omitted in the interest of brevity.

User code executes in the \texttt{TaskRunning} state. Several operations, such
as querying the \texttt{POST} and \texttt{GET} variables, are available in this state, alongside
functions to output data to the web page and append data to the response
headers. It is important to note that at this stage nothing is written to the
page, with the \texttt{output} and \texttt{addHeader} functions instead
modifying the \texttt{CGIInfo} record. This data may then be printed at the end of the
program's execution, in accordance with the resource usage protocol.

After the user code has finished execution, control returns to the library
code. At this point, the state transitions to TaskCompleted, and the headers
are written.  Finally, the headers and content are written which completes the
process. Since we parameterise the resource over a state, we may ensure that
certain operations only happen in a particular prescribed order.

In \idris{}, types are first-class, meaning that they may be treated like other
terms in computations. We may therefore define the following type synonym, used
within the CGI section of the framework to denote an effectful CGI program: 

\begin{SaveVerbatim}{cgiprog}

CGIProg : List EFFECT -> Type -> Type
CGIProg effs a = 
  Eff IO (CGI (InitialisedCGI TaskRunning) :: effs) a

\end{SaveVerbatim}
\useverb{cgiprog}

\noindent
This is then passed, along with initial values for other effects that the user
may wish to use, to the \texttt{runAction} function, which invokes the \texttt{RunAction}
operation and executes the user-specified action.
%
A simple ``Hello, world!'' program would be defined as follows:

\begin{SaveVerbatim}{hworldcgi}

module Main
import Cgi

sayHello : CGIProg [] ()
sayHello = output "Hello, world!"

main : IO ()
main = runCGI [initCGIState] sayHello

\end{SaveVerbatim}
\useverb{hworldcgi}
% =================================================

Here, \texttt{output} is a function which appends some output to the CGI output buffer, which is later written to the page.

\subsection{Database access with SQLite}

SQLite\footnote{\texttt{http://www.sqlite.org}} is a lightweight SQL database
engine often used as simple, structured storage for larger applications. We
make use of SQLite to demonstrate a resource usage protocol for database access due to its simplicity, although we envisage that these
concepts would be applicable to more complex database management systems. 

The creation, preparation and execution of SQL queries has a specific usage
protocol, with several possible points of failure. Failure is handled in
traditional web applications by the generation of exceptions, which may be
handled in the program.  Handling such exceptions is often optional, however,
and in some cases unhandled errors may cause a deployed web application to
display an error to the user. Such errors can be used to determine the
structure of an insecure SQL query, and are often used by attackers to
determine attack vectors for SQL injection attacks.

\begin{SaveVerbatim}{databaseeff}
{-                           { Input resource type }    { Output resource type }             { Value }           -}
data Sqlite : Effect where
  OpenDB            : DBName -> 
                      Sqlite ()                         (Either () SQLiteConnected)          (Either SQLiteCode ())
  CloseDB           : Sqlite (SQLiteConnected)          ()                                   ()
  PrepareStatement  : QueryString -> 
                      Sqlite (SQLiteConnected)          (Either (SQLitePSFail) 
                                                          (SQLitePSSuccess Binding))         (Either SQLiteCode ())
  BindInt           : ArgPos -> Int -> 
                      Sqlite (SQLitePSSuccess Binding)  (SQLitePSSuccess Binding)            ()
  FinishBind        : Sqlite (SQLitePSSuccess Binding)  (Either SQLiteFinishBindFail
                                                          (SQLitePSSuccess Bound))           (Maybe BindError)
  ExecuteStatement  : Sqlite (SQLitePSSuccess Bound)    (Either (SQLiteExecuting InvalidRow)
                                                          (SQLiteExecuting ValidRow)         StepResult
  RowStep           : Sqlite (SQLiteExecuting ValidRow) (Either (SQLiteExecuting InvalidRow)
                                                          (SQLiteExecuting ValidRow))        StepResult
  GetColumnText     : Column -> 
                      Sqlite (SQLiteExecuting ValidRow) (SQLiteExecuting ValidRow)           String
                      
  CleanupPSFail     : Sqlite (SQLitePSFail)             ()                                   ()
  CleanupBindFail   : Sqlite (SQLiteFinishBindFail)     ()                                   ()

\end{SaveVerbatim}

\begin{figure*}[t]
\begin{center}
\useverb{databaseeff}
\end{center}
\caption{Database Effect}
\label{fig:dbeffect}
\end{figure*}

%TODO: Fix up the labels
\begin{figure}
\centering
\scalebox{0.8}{
\begin{tikzpicture}[>=latex]
  \tikzset{every loop/.style={min distance=10mm,looseness=5}}
  \tikzstyle{state} = [draw, very thick, fill=white, rectangle, minimum height=3em, minimum width=7em, align=center, node distance=7em, font={\sffamily\bfseries}]
  
  \tikzstyle{stateEdgePortion} = [black,thick];
  \tikzstyle{stateEdge} = [stateEdgePortion,->];
  \tikzstyle{edgeLabel} = [pos=0.5, font={\sffamily\small}];

  \node[initial,state] (A)              {Uninitialised};
  \node[state]         (B) [below of=A] {SQLiteConnected};
  \node[state]         (C) [below of=B] {SQLitePSSuccess Binding};
  \node[state]         (D) [below of=C] {SQLitePSSuccess Bound};
  \node[state]         (E) [below left of=D, xshift=-4em] {SQLiteExecuting \\ ValidRow};
  \node[state]         (F) [below right of=D, xshift=4em] {SQLiteExecuting \\ InvalidRow};

  \path (A) edge[stateEdge]   node[edgeLabel, xshift=2.5em] {\texttt{openDB}} (B)
        (B) edge[stateEdge, bend right=80]   node[edgeLabel, text width=2cm, xshift=4.5em] {\texttt{closeDB, connFail}} (A)
        (B) edge[stateEdge]   node[edgeLabel, xshift=5em] {\texttt{prepareStatement}} (C)
        (C) edge[stateEdge]   node[edgeLabel, xshift=3.25em, yshift=-0.75em] {\texttt{finishBind}} (D)
        (D) edge[stateEdge]   node[edgeLabel, text width=2cm, xshift=-4.5em, yshift=-1em] {\texttt{executeStatement, nextRow}} (F)
        (D) edge[stateEdge]   node[edgeLabel, xshift=3em] {} (E)
        (E) edge[stateEdge, bend left=35]   node[edgeLabel, xshift=3em, yshift=-5em] {\texttt{finaliseValid}} (A)
        (F) edge[stateEdge, bend right=35]   node[edgeLabel, xshift=-3em, yshift=-5em] {\texttt{finaliseInvalid}} (A)
        (E) edge[stateEdge]   node[edgeLabel, yshift=-1em] {\texttt{nextRow}} (F)
        (F) edge[stateEdge, bend left=35]   node[edgeLabel, yshift=-1em] {\texttt{reset}} (E)
        (E) edge[stateEdge, loop below]   node[edgeLabel] {\texttt{nextRow}} (E);
        
\end{tikzpicture}
}
\caption{Database Resource Usage Protocol}
\label{fig:sqlitestates}
\end{figure}

Figure~\ref{fig:sqlitestates} shows a resource usage protocol for database
access, which we have implemented for the SQLite library. Although some
additional states are used to capture failing computations, these are omitted
from the diagram.
The effect implementation is given in Figure \ref{fig:dbeffect}.
%
There are three main phases involved in the usage of the SQLite protocol:
connection to the database, preparation of a query, and execution of the query.
We define several resources to encapsulate the state at any given point during the protocol.

We first define the \texttt{SQLiteConnected} resource, which signifies that a
successful connection has been made to the database. This resource contains a
pointer to the database structure which is used in further computations.

\begin{SaveVerbatim}{sqliteconnected}

data SQLiteConnected : Type where
  SQLConnection : ConnectionPtr -> SQLiteConnected
  
\end{SaveVerbatim}  
\useverb{sqliteconnected}

\noindent
We secondly define resource types to capture success and failure states
of binding a prepared statement:
\texttt{SQLitePSSuccess}, 
\texttt{SQLitePSFail}, and \texttt{SQLiteFinishBindFail}. The types are
declared as follows (we leave the definitions abstract):

\begin{SaveVerbatim}{sqliteconnected}

data BindStep = Binding | Bound
data SQLitePSSuccess : BindStep -> Type where
data SQLitePSFail : Type where
data SQLiteFinishBindFail : Type where

\end{SaveVerbatim}

%  SQLitePS       : ConnectionPtr -> 
%                   StmtPtr -> SQLitePSSuccess a
%  SQLiteBindFail : ConnectionPtr -> 
%                   StmtPtr -> 
%                   BindError -> SQLitePSSuccess a
%data SQLitePSFail : Type where
%  PSFail : ConnectionPtr -> SQLitePSFail
%
%data SQLiteFinishBindFail : Type where
%  SQLiteFBFail : ConnectionPtr -> StmtPtr -> 
%                 SQLiteFinishBindFail

\useverb{sqliteconnected}

\noindent
The \texttt{SQLitePSSuccess} resource indicates that a prepared statement has been correctly created by the underlying library, given a query string. The resource is parameterised by the \texttt{BindStep} type, which indicates whether the binding process has been completed. Should the creation of a prepared statement fail, the resource will be set to \texttt{SQLitePSFail}. 

%To avoid having to explicitly check for failures each time a variable is bound to an argument in the prepared statement, any failures are recorded by constructing the resource with the \texttt{SQLiteBindFail} constructor. Should any binding operations be called whilst in this state, the operations will not be passed to the underlying library. After the binding stage has completed, the \texttt{FinishBind} operation is called. This operation inspects the resource and either transitions to the \texttt{SQLitePSSuccess Bound} state, indicating that all bind operations were successful, or to the \texttt{SQLiteFinishBindFail} state, should a binding operation have been unsuccessful. 

SQLite operates by loading database rows into a buffer, the contents of which may then be accessed through several column access functions. After all rows returned by a query have been processed, no further calls to fetch more rows may be made. Additionally, no calls to column access functions may be made whilst there is no row being currently processed. 

Through the use of dependent types, we may encode these invariants statically, using the \texttt{SQLiteExecuting} resource. This is parameterised by the \texttt{ExecutionResult} data type, which indicates whether there is currently a valid row in the buffer.

\begin{SaveVerbatim}{sqliteexecuting}

data ExecutionResult = ValidRow
                     | InvalidRow

data SQLiteExecuting : ExecutionResult -> Type where
  SQLiteE : ConnectionPtr -> 
            StmtPtr -> SQLiteExecuting a
  
\end{SaveVerbatim}
\useverb{sqliteexecuting}
\noindent
Column access functions may only be called when there is a valid row in the
buffer, as signified by the input resources:

\begin{SaveVerbatim}{getcoltext}

GetColumnText : Column -> 
                Sqlite (SQLiteExecuting ValidRow) 
                       (SQLiteExecuting ValidRow)
                       String
                         
\end{SaveVerbatim}
\useverb{getcoltext}

\noindent
In order to provide the necessary static guarantee that there is a valid row to
process in the buffer, we make use of the \texttt{if\_valid} construct, as described in Section ~\ref{rp-effects}. The
\texttt{nextRow} function will either fetch a row into the buffer, or indicate
that a row could not be loaded, as shown by its type:

\noindent
\begin{SaveVerbatim}{nextrow}

nextRow : EffM IO [SQLITE (SQLiteExecuting ValidRow)] 
         [SQLITE (Either (SQLiteExecuting InvalidRow)
                 (SQLiteExecuting ValidRow))] StepResult

\end{SaveVerbatim}
\useverb{nextrow}

\noindent
The \texttt{if\_valid} construct provides failure checking functionality, allowing different operations to be performed depending on whether or not a row was successfully fetched for processing.
%If a failure happens at any point during the computation, the resource is
%updated to reflect the failure: if, for example, the library failed to create a
%connection to the database, the resource value would be updated to
%\texttt{OpenConn InvalidConn}. At this point, no further side-effecting
%requests are made to the underlying SQLite library, in order to ensure safety.
%The \texttt{connFail}, \texttt{stmtFail}, \texttt{bindFail} and
%\texttt{executeFail} utility functions allow for failures, once detected, to be
%handled by executing the appropriate sequence of state transition functions to
%dispose of any open resources and return to the initial protocol state. 

%Queries are evaluated through one or more calls to the \texttt{nextRow} function, which either executes an update statement or returns the next row of a result set. 
%SQL queries are evaluated in SQLite upon a call to the C library function
%\texttt{sqlite3\_step()}. In the case that a statement returns a result set,
%each subsequent call retrieves another row for processing using a column access
%function. Once all rows have been retrieved, the library returns
%\texttt{SQLITE\_DONE}, meaning that no further calls should be made without
%resetting the function. We encapsulate this requirement through the
%\texttt{StepResult} data type within the \texttt{ExecutingStmt} constructor. 

%\begin{SaveVerbatim}{stepres}

%data StepResult = Unstarted    | StepFail
%                | StepComplete | NoMoreRows

%\end{SaveVerbatim}
%\useverb{stepres}

By incorporating pointers to open connections and prepared statements into the
resource associated with the effect, we introduce a further layer of
abstraction, which hides implementation details from the developer and
encourages less error-prone code. 

\subsubsection{Example}

%Programs making use of the DSL should look familiar to developers even without
%a background in functional programming. 
To demonstrate the 
library, we return to the previous example of selecting the names and addresses
of all staff working in a given department. 
%Since the
%\texttt{Effects} library overloads the bind operator, we may make use of
%\texttt{do}-notation, facilitating the usage of an imperative style.
We define a function \texttt{textSel} of type:

\noindent
\begin{SaveVerbatim}{sqleffty}

String -> Eff IO [SQLITE ()] 
           (Either QueryError (List (String, String)))

\end{SaveVerbatim}
\useverb{sqleffty}

\noindent
The program will be run in \texttt{IO}, and starts and finishes with no active
resources.  It returns either a list of \texttt{(String, String)} pairs,
representing names and addresses in the database, or an error.
%This example is shown in Figure
%\ref{fig:testsel}.

\begin{SaveVerbatim}{testsel}
testSelect : String -> Eff IO [SQLITE ()] 
             (Either QueryError (List (String, String)))
testSelect dept = do
  db_res <- openDB "people.db"
  if_valid then do
    let sql = "SELECT name, address FROM `staff` 
                    WHERE dept = ?;"
    sql_prep_res <- prepareStatement sql
    if_valid then do 
      bindText 1 dept
      bind_res <- finishBind
      if_valid then do
        executeStatement
        results <- collectResults
        finaliseInvalid
        closeDB
        Effects.pure $ Right results
      else do
        cleanupBindFail
        Effects.pure $ Left (getBindError bind_res)
    else do
      cleanupPSFail
      Left $ getQueryError sql_prep_res
  else do 
    Effects.pure $ Left (getQueryError db_res)
\end{SaveVerbatim}

\begin{figure}[h]
\useverb{testsel}
\caption{Example SQL program}
\label{fig:testsel}
\end{figure}

The program initially attempts to open a connection to the \texttt{people.db}
database. At this point, since the \texttt{OpenDB} operation has been invoked,
the program transitions to the \texttt{ConnectionOpened} state. The
\texttt{openDB} function returns either an error code, if the connection fails, or a unit type should the connection succeed, as shown in Figure \ref{fig:sqlitestates}.

A call to \texttt{prepareStatement} attempts to create a prepared statement,
and a subsequent call to \texttt{beginExecution} allows data to be retrieved
from the database.
%
\texttt{CollectResults} operates on the row currently held in the buffer.
Firstly, the function checks
that there is a valid row in the buffer, and if so uses the
\texttt{getColumnText} function to retrieve the data
from the database. This function is then called recursively until there are no
more rows to process.

\begin{SaveVerbatim}{collectresty}

collectResults :
  EffM IO [SQLITE (Either (SQLiteExecuting InvalidRow) 
                          (SQLiteExecuting ValidRow))] 
          [SQLITE (SQLiteExecuting InvalidRow)] 
          (List (String, String))
\end{SaveVerbatim}

\begin{SaveVerbatim}{collectres}
collectResults = 
  if_valid then do name <- getColumnText 1
                   addr <- getColumnText 2
                   step_res <- nextRow
                   xs <- collectResults
                   return $ (name, addr) :: xs
  else return []

\end{SaveVerbatim}

\noindent
\useverb{collectresty}

\noindent
\useverb{collectres}

\noindent
Using this, we can build a function to execute a full query:

\begin{SaveVerbatim}{execsel}

executeSelect : 
  String -> String -> List (Int, DBVal) -> 
  (Eff IO [SQLITE (SQLiteExecuting ValidRow)] 
               (List DBVal)) -> 
  Eff IO [SQLITE ()] (Either QueryError ResultSet)
 
\end{SaveVerbatim}
\useverb{execsel}

\noindent
This returns either an error, or a set of results, defined
as follows:

\begin{SaveVerbatim}{dbvals}

ResultSet : Type
ResultSet = List (List DBVal)

data DBVal = DBInt Int     | DBText String
           | DBFloat Float | DBNull

\end{SaveVerbatim}
\useverb{dbvals}


% =================================================
\subsection{A Simple Session Handler}
Larger web applications require persistent state across separate
requests. This can be achieved using a \textit{session},
in which a cookie is set on the remote host containing a unique session ID,
which is used to retrieve data. In this section, we describe
a simple session handler, and the resource protocol
involved. 

The \texttt{Effects} library allows for composition of individual, fine-grained
effects. By combining the \texttt{CGI} and \texttt{SQLite} components, we can
construct a simple session handler to provide a notion of state across separate
web requests. 
We implement this with a SQLite database containing tables for storing
session keys and expiry dates, along with the data associated with
the session.

%We implement this with a SQLite database, containing two tables:
%\texttt{session}, which stores session keys and their associated expiry dates,
%and \texttt{sessiondata}, which contains the data associated with each session.
%A datum associated with the session is described as a tagged union containing
%one of the primitive types \texttt{String}, \texttt{Bool} or \texttt{Int},
%which is serialised alongside a type tag for storage in the database.
%TODO: Fix up the labels

\begin{SaveVerbatim}{sessioneff}
{-                        { Input resource type }            { Output resource type }   { Value }        -}

data Session : Effect where
  LoadSession           : SessionID -> 
                          Session (SessionRes SessionClosed) (SessionRes SessionOpen)   (Maybe SessionData)
  UpdateSession         : SessionData -> 
                          Session (SessionRes SessionOpen)   (SessionRes SessionOpen)   ()
  CreateSession         : SessionData -> 
                          Session (SessionRes SessionClosed) (SessionRes SessionOpen)   (Maybe SessionID)
  DeleteSession         : Session (SessionRes SessionOpen)   (SessionRes SessionClosed) Bool 
  WriteToDB             : Session (SessionRes SessionOpen)   (SessionRes SessionClosed) Bool
  DiscardSessionChanges : Session (SessionRes SessionOpen)   (SessionRes SessionClosed) ()
  GetSessionID          : Session (SessionRes SessionOpen)   (SessionRes SessionOpen)   (Maybe SessionID)
  GetSessionData        : Session (SessionRes SessionOpen)   (SessionRes SessionOpen)   (Maybe SessionData)
\end{SaveVerbatim}

\begin{figure*}[t]
\begin{center}
\useverb{sessioneff}
\end{center}
\caption{Session Effect}
\label{fig:sessioneffect}
\end{figure*}

\begin{figure}[htpb!]
\centering
\scalebox{0.8}{
\begin{tikzpicture}[>=latex]
  \tikzstyle{state} = [draw, very thick, fill=white, rectangle, minimum height=3em, minimum width=7em, node distance=7.5em, font={\sffamily\bfseries}]
  
  \tikzstyle{stateEdgePortion} = [black,thick];
  \tikzstyle{stateEdge} = [stateEdgePortion,->];
  \tikzstyle{edgeLabel} = [pos=0.5, font={\sffamily\small}];

  \node[initial,state] (A)              {SessionClosed};
  \node[state]         (B) [below of=A] {SessionOpen};

  \path (A) edge[stateEdge]   node[edgeLabel, xshift=-1cm, text width=3cm] {\texttt{loadSession, createSession}} (B)
        (B) edge[stateEdge, bend right=80]   node[edgeLabel, xshift=6em, text width=3cm] {\texttt{writeToDB, discardSessionChanges, deleteSession}} (A)
            edge[stateEdge, loop left]   node[edgeLabel] {\texttt{updateSession}} (B);
\end{tikzpicture}
}
\caption{Session Handler Resource Usage Protocol}
\label{fig:sessionstates}
\end{figure}

Figure ~\ref{fig:sessionstates} shows the resource usage protocol associated
with the session handler, and Figure \ref{fig:sessioneffect} the corresponding
algebraic effect. In this application, there are two states:
In \texttt{SessionClosed}, the user may load or create a
session.
In \texttt{SessionOpen}, the user may update the
representation of the session in memory, serialise the session and write it to
the database, or delete the session and invalidate the user's session key. 
These two states ensure that changes are explicitly either
written or discarded, eliminating the possibility of a developer updating the
session but neglecting to commit it to persistent storage. This, of course, is
under the assumption that the process exits cleanly: we attempt to facilitate
this by writing total functions where possible.

Much like the SQLite effect, we encapsulate failure by reflecting it in the
resource associated with the effect. 

\begin{SaveVerbatim}{sstep}

data SessionStep = SessionClosed | SessionOpen
\end{SaveVerbatim}

\begin{SaveVerbatim}{sres}
data SessionRes : SessionStep -> Type where
  InvalidSession : SessionRes s  
  ValidSession   : SessionID -> 
                   SessionData -> 
                   SessionRes s

\end{SaveVerbatim}

\useverb{sstep}

\useverb{sres}

%The \texttt{SessionRes} data type is parameterised over the current state,
%which determines which operations may be performed, and has two constructors:
%\texttt{InvalidSession} and \texttt{ValidSession}. If an operation such as
%creating a new session fails, no further side-effecting calls will be made, in
%turn preserving integrity. 

% =================================================